\documentclass[a4paper]{article}

\usepackage[english]{babel}
\usepackage[utf8x]{inputenc}
\usepackage{amsmath}
\usepackage{graphicx}
\usepackage[colorinlistoftodos]{todonotes}

\title{Convex Optimization PSET 1}
\author{Rafael Cosman}

\begin{document}
\maketitle

\section{Problem 2.5}
The distance between two parallel hyperplanes is $|b_{2} - b_{1}| / ||a||$.

Reasoning:
Consider the first hyperplane. We will compute its distance from the origin.

The closest point on the plane to the origin will be $b/||a||^2$

\section{Problem 2.7}
We wish to describe $x| ||x-a||_2\leq||x-b||_2$ as an inequality of the form $c^Tx \leq d$

Let us manipulate $||x-a||_2 \leq ||x-b||_2$

$||x-a||_2^2 \leq ||x-b||_2^2$

$(x-a)^T(x-a) \leq (x-b)^T(x-b)$

\section{Problem 2.12}
\subsection{Subproblem a: slab}
Convex. Because it’s the intersection of two half-spaces, which we can assume are convex.
\subsection{Subproblem b: rectangle}
Convex. Because it’s the intersection of $2n$ half-spaces, which we assume are convex.
\subsection{Subproblem c: wedge}
Convex. Intersection of half-spaces.
\subsection{Subproblem d: set of points closer to a point than to a set}
Convex. This is just the intersection of n half-spaces, assuming that there are n elements of the set. The reasoning behind this is the same as that in 2.7.
\subsection{Subproblem e: set of points closer to set $S$ than to set $T$}
Not Convex. I will provide an example in $R$. Let $s = {-2, 2}$. Let $T = {0}$. The points closer to $S$ than to $T$ are all ${x | x < -1 or x > 1}$, which is trivially not a convex set, because it is disjoint.
\subsection{Subproblem f}
Convex. Take $S_1$, translate it by each element of $S_2$ and take the intersection of those.
\subsection{Subproblem g}
Convex. This goes from a tiny ellipse (theta near 0) up to a plane at (theta=1). If theta could exceed 1, the set would be concave.


\section{Problem 2.15}

First we will prove that $Ef(x) <= Eg(x)$ is a convex set regardless of $f$ and $g$. This will allow us to solve a bunch of problems very easily later.

We rewrite this as $bTp <= cTp$,
where $b$ is a vector corresponding to $f(a[i])$ for all $i$
and $c$ is a vector corresponding to $g(a[i])$ for all $i$

We can manipulate this to be (c-b)Tp >= 0

The x that satisfy this inequality is a convex set, because this is just the equation for a polygon.

Thus for all f and g, Ef(x) <= Eg(x). We will refer to this as theorem 1.
a:
Convex. This is a specific case of theorem 1, where f(x) is 
b:
c:
Convex. This is a specific case of theorem 1, where f(x) is |x3| and g(x) is alpha*|x|
d:
Convex. This is a specific case of theorem 1, where f(x) is x2 and g(x) is alpha
e:
Convex. This is a specific case of theorem 1, where f(x) is -x2 and g(x) is -alpha. (Flipping both signs causes the inequality to reverse.)
f:
Not Convex. Proof by counterexample. Let alpha be 0. This yields the set of probability distributions with zero variance. This set has n elements. The first element has p1 = 1 and all other p = 0. The second element has p2 = 1 and all other p = 0. Etc.

These points are at the vertices of the simplex. The set is clearly not convex.
g:
Convex. It’s complement is a concave set.
h:
Convex. This is a specific case of theorem 1, where f(x) is 1 for all x <= alpha, and 0 for all x > alpha. g(x) is .25.
i:


\section{Problem 2.23}
We will work in R2, and use the convention of representing each element of $R^2$ as $(x, y)$
First set: $epi(y=1/x)$
Second set: $y=0$

Both sets are closed.

The key to why they are not strictly separable is that for any line y=epsilon, the First set will include points below the line (for example, the point at $(2/epsilon, epsilon/2)$)

\section{Problem 2.28}
\subsection{$n = 1$}
In this case, $x1 > 0$ is the only condition necessary.

\subsection{$n = 2$}
As discussed in class,
$x_1 \geq 0$
$x_3 \geq 0$
$x_1x_3 - x_2x_2 \geq 0$

\subsection{$n = 3$}
Firstly, all of the elements on the diagonal must be non-negative:

$x_1 \geq 0$

$x_4 \geq 0$

$x_4 \geq 0$


All of the 2x2 determinants must be non-negative:

$x_1x_4 - x_2x_2 \geq 0$

$x_1x_6 - x_3x_3 \geq 0$

$x_4x_6 - x_5x_5 \geq 0$


The 3x3 determinant of the whole matrix must be non-negative:

$(x_1x_4x_6 + x_2x_5x_3 + x_2x_5x_3) - (x_3x_4x_3 + x_5x_5x_1 + x_6x_2x_2) \geq 0$

These conditions are necessary and sufficient for the matrix to be positive semidefinite.

\end{document}